\documentclass[14pt]{extarticle}
\usepackage[utf8]{inputenc}
\usepackage[T1]{fontenc}
\usepackage[spanish,es-lcroman]{babel}
\usepackage{amsmath}
\usepackage{amsthm}
\usepackage{physics}
\usepackage{tikz}
\usepackage{float}
\usepackage[autostyle,spanish=mexican]{csquotes}
\usepackage[per-mode=symbol]{siunitx}
\usepackage{gensymb}
\usepackage{multicol}
\usepackage{enumitem}
\usepackage[left=2.00cm, right=2.00cm, top=2.00cm, 
     bottom=2.00cm]{geometry}
\usepackage{Estilos/ColoresLatex}

\newcommand{\textocolor}[2]{\textbf{\textcolor{#1}{#2}}}

%\renewcommand{\questionlabel}{\thequestion)}
\decimalpoint
\sisetup{bracket-numbers = false}

\title{\vspace*{-2cm} Ejercicio Recuperación Práctica 3 - Física 1\vspace{-5ex}}
\date{\today}

\begin{document}
\maketitle

\section{Ejercicios a resolver.}

\subsection{Ejercicio 1.}

En una persecución policial, el automóvil en fuga inicia con los siguientes desplazamientos:
\begin{enumerate}
\item $d_{1} = \SI{120}{\meter}, \theta = \ang{10}$
\item $d_{2} = \SI{900}{\meter}, \theta = \ang{100}$
\item $d_{3} = \SI{700}{\meter}, \theta = \ang{200}$
\item $d_{4} = \SI{500}{\meter}, \theta = \ang{300}$
\end{enumerate}
Si una patrulla está en la posición donde inició el automóvil la fuga. ¿Cuál es el desplazamiento que tiene que realizar la patrulla par encontrarse con el automóvil que se dio a la fuga?
\\
Determina con el método analítico, el valor del desplazamiento resultante y su ángulo con respecto al eje $x$ positivo.

\subsection{Ejercicio 2.}

Del siguiente sistema de vectores, calcula el vector resultante y el ángulo del mismo. Presenta las tablas correspondientes a los vectores y de sus componentes.
\begin{figure}[H]
\centering
\begin{tikzpicture}[scale=1]
\draw [-stealth] (-10, 0) -- (5, 0) node [above, pos=1] {\small{$x$}};
\draw [-stealth] (0, -7) -- (0, 7) node [left, pos=1] {\small{$y$}};
\draw [-stealth, thick, color=red] (0, 0) -- (3.87, 2.32 ) node [above, pos=1] {\small{$\vb{v}_{1} = \SI{59}{\meter\per\second}$}};
\draw [color=red] (0.5, 0) arc(0:39:0.5);
\node at (2.5, 0.3) [color=red] {\small{$\theta_{1} = \ang{39}$}};

\draw [-stealth, thick, color=regalia] (0, 0) -- (2.16, -6.65) node [above, near end, rotate=-72] {\small{$\vb{v}_{2} = \SI{70}{\meter\per\second}$}};
\draw [color=regalia] (0.5, 0) arc(360:288:0.5);
\node at (2.5, -0.8) [color=regalia] {\small{$\theta_{2} = \ang{72}$}};

\draw [-stealth, thick, color=rosewood] (0, 0) -- (-9.45, -3.25) node [below, near end, rotate=19] {\small{$\vb{v}_{3} = \SI{100}{\meter\per\second}$}};
\draw [color=rosewood] (-0.5, 0) arc(180:199:0.5);
\node at (-3.5, -0.5) [color=rosewood] {\small{$\theta_{3} = \ang{19}$}};

\draw [-stealth, thick, color=ultramarine] (0, 0) -- (-5.2, 0) node [above, near end] {\small{$\vb{v}_{4} = \SI{52}{\meter\per\second}$}};

\draw [-stealth, thick, color=persimmon] (0, 0) -- (-1.04, 6.61) node [near end, xshift=-0.2cm, yshift=-0.5cm, rotate=-81] {\small{$\vb{v}_{5} = \SI{67}{\meter\per\second}$}};
\draw [thick, color=persimmon] (-0.5, 0) arc(180:99:0.5);
\node at (-2.3, 4) [rotate=-81, color=persimmon] {\small{$\theta_{5} = \ang{81}$}};
\draw [-stealth, color=persimmon] (-2, 2.6) -- (-0.6, 0.5);
\end{tikzpicture}
\end{figure}
  

\end{document}