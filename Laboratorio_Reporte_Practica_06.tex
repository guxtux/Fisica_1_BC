\documentclass[14pt]{extarticle}
\usepackage[utf8]{inputenc}
\usepackage[T1]{fontenc}
\usepackage[spanish,es-lcroman]{babel}
\usepackage{amsmath}
\usepackage{amsthm}
\usepackage{physics}
\usepackage{tikz}
\usepackage{float}
\usepackage[autostyle,spanish=mexican]{csquotes}
\usepackage[per-mode=symbol]{siunitx}
\usepackage{gensymb}
\usepackage{multicol}
\usepackage{enumitem}
\usepackage[left=2.00cm, right=2.00cm, top=2.00cm, 
     bottom=2.00cm]{geometry}
\usepackage{Estilos/ColoresLatex}
\usepackage{makecell}

\newcommand{\textocolor}[2]{\textbf{\textcolor{#1}{#2}}}
\DeclareSIUnit[number-unit-product = {\,}]\cal{cal}

%\renewcommand{\questionlabel}{\thequestion)}
\decimalpoint
\sisetup{bracket-numbers = false}

\title{\vspace*{-2cm} Práctica 6 Cálculo del valor de $g$ - Física 1\vspace{-5ex}}
\date{\today}

\begin{document}
\maketitle

La elaboración y entrega del reporte en físico es INDIVIDUAL. Genera una portada en donde indicarás tu nombre completo.

Responde cada una de las preguntas del apartado \enquote{Investiga y escribe brevemente} de la Práctica 6 - Cálculo del valor de $g$ del Manual de Prácticas.

\textbf{Objetivo: } Determinar de manera experimental con un péndulo el valor de la aceleración debida a la gravedad. 

Actividades a realizar.
\begin{enumerate}
\item Se han montado un soporte universal con una varilla, en la que está atado un hilo con una masa, así como una pinza con un hilo y otra masa, la pinza está sujeta del techo.
\item Deberás de medir el período de oscilación del péndulo para un desplazamiento angular determinado, es decir, deberás de anotar el tiempo que tarda el péndulo en completar 5 oscilaciones completas.
\item Repite 5 ocasiones el registro del período del péndulo, con el mismo desplazamiento angular.

Longitud hilo $L_{1}$: 
\begin{table}[H]
\centering
\begin{tabular}{| c | c |} \hline
Medición & Período [\unit{\second}] \\ \hline
$t_{1}$ &  \\ \hline
$t_{2}$ &  \\ \hline
$t_{3}$ &  \\ \hline
$t_{4}$ &  \\ \hline
$t_{5}$ &  \\ \hline    
\end{tabular}
\end{table}
Calcula el promedio del período de oscilación del péndulo: $T_{\text{promedio1}}$: \rule{2cm}{0.3mm}
\item Completa la misma tabla de mediciones del período de oscilación de los otros dos péndulos, toma en cuenta que las longitudes del hilo son distintas:

\begin{minipage}{0.4\linewidth}
Longitud hilo $L_{2}$: 
\begin{table}[H]
\centering
\begin{tabular}{| c | c |} \hline
Medición & Período [\unit{\second}] \\ \hline
$t_{1}$ &  \\ \hline
$t_{2}$ &  \\ \hline
$t_{3}$ &  \\ \hline
$t_{4}$ &  \\ \hline
$t_{5}$ &  \\ \hline    
\end{tabular}
\end{table}
$T_{\text{promedio2}}$: \rule{2cm}{0.3mm}
\end{minipage}
\hspace{0.5cm}
\begin{minipage}{0.4\linewidth}
Longitud hilo $L_{3}$: 
\begin{table}[H]
\centering
\begin{tabular}{| c | c |} \hline
Medición & Período [\unit{\second}] \\ \hline
$t_{1}$ &  \\ \hline
$t_{2}$ &  \\ \hline
$t_{3}$ &  \\ \hline
$t_{4}$ &  \\ \hline
$t_{5}$ &  \\ \hline    
\end{tabular}
\end{table}
$T_{\text{promedio3}}$: \rule{2cm}{0.3mm}
\end{minipage}
\item Una vez completadas las tres tablas, revisa que cada integrante del equipo tenga los datos.
\item Corta el hilo de cada péndulo y mide su longitud primero con la regla de madera, luego con el flexómetro y anota los datos en la siguiente tabla:
\begin{table}[H]\centering
\begin{tabular}{| c | c | c |} \hline
Hilo & Regla (\unit{\meter}) & Flexómetro (\unit{\meter}) \\ \hline
$L_{1}$ & & \\ \hline
$L_{2}$ & & \\ \hline
$L_{3}$ & & \\ \hline
\end{tabular}
\end{table}
\item La siguiente expresión corresponde al valor del período de oscilación de un péndulo $T$, y donde $L$ es la longitud del hilo:
\begin{align*}
T = 2 \, \pi \sqrt{\dfrac{L}{g}}
\end{align*}
Despeja el valor de $g$.
\item Con el despeje que has hecho, calcula el valor de $g$ para cada péndulo, donde ya conoces el valor de la longitud tanto con la regla, como con el flexómetro, y período de oscilación, calcula el error relativo considerando el valor de \break \hfill $g = \SI{9.81}{\meter\per\square\second}$, como valor exacto.

\textbf{Longitud del hilo medida con la regla de madera.}
\begin{table}[H]
    \centering
\begin{tabular}{| c | p{3cm} | p{3cm} | p{3cm} | c |} \hline
Péndulo & \makecell{Longitud} & \makecell{Período} & \makecell{$g$} & Error relativo \\ \hline
$1$ & $L_{1} = $ & $T_{1} = $ & & \\ \hline
$2$ & $L_{2} = $ & $T_{2} = $ & & \\ \hline
$3$ & $L_{3} = $ & $T_{3} = $ & & \\ \hline
\end{tabular}
\end{table}

\textbf{Longitud del hilo medida con el flexómetro.}
\begin{table}[H]
    \centering
\begin{tabular}{| c | p{3cm} | p{3cm} | p{3cm} | c |} \hline
Péndulo & \makecell{Longitud} & \makecell{Período} & \makecell{$g$} & Error relativo \\ \hline
$1$ & $L_{1} = $ & $T_{1} = $ & & \\ \hline
$2$ & $L_{2} = $ & $T_{2} = $ & & \\ \hline
$3$ & $L_{3} = $ & $T_{3} = $ & & \\ \hline
\end{tabular}
\end{table}

\item ¿Para qué péndulo se aproxima mejor el valor de $g$. ¿Por qué? Justifica tu respuesta.
\item ¿Para qué péndulo se el valor de $g$ tiene el mayor error. ¿Por qué? Justifica tu respuesta.
\item ¿Cómo podrías mejorar la aproximación del valor de $g$. Justifica tu respuesta.
\end{enumerate}

\textbf{Nota importante: } Esta hoja firmada por el Profesor deberás de incluirla en tu reporte, \textocolor{red}{la hoja NO ES EL REPORTE}, deberás de integrar el apartado inicial, el objetivo, así como una descripción detallada de cada punto que realizaste, además de responder las preguntas que en esta hoja de indican.

\end{document}