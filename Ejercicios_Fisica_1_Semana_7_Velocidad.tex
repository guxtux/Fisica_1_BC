\documentclass[14pt]{extarticle}
\usepackage[utf8]{inputenc}
\usepackage[T1]{fontenc}
\usepackage[spanish,es-lcroman]{babel}
\usepackage{amsmath}
\usepackage{amsthm}
\usepackage{physics}
\usepackage{tikz}
\usepackage{float}
\usepackage[autostyle,spanish=mexican]{csquotes}
\usepackage[per-mode=symbol]{siunitx}
\usepackage{gensymb}
\usepackage{multicol}
\usepackage{enumitem}
\usepackage[left=2.00cm, right=2.00cm, top=2.00cm, 
     bottom=2.00cm]{geometry}
\usepackage{Estilos/ColoresLatex}

\newcommand{\textocolor}[2]{\textbf{\textcolor{#1}{#2}}}

%\renewcommand{\questionlabel}{\thequestion)}
\decimalpoint
\sisetup{bracket-numbers = false}

\title{\vspace*{-2cm} Ejercicios de Velocidad - Física 1\vspace{-5ex}}
\date{\today}

\begin{document}
\maketitle

\section{Ejercicios a cuenta}

Estos ejercicios otorgarán hasta $6$ puntos a la Evaluación Continua.

La entrega se hará vía Teams en asignación, teniendo como plazo el día domingo 9 de julio a las 8 pm.

Cada ejercicio vale $1$ punto, siempre y cuando esté correcto. Se otorgará una parte proporcional en caso de tener desarrollo detallado pero el resultado no sea el esperado.

La solución al ejercicio deberá de desarrollarse de manera completa: datos, expresión, sustitución, manejo de unidades y resultado. En caso de presentar solo el resultado, es decir, sin desarrollo, aunque el resultado sea el correcto, la solución no aportará puntaje.

Anota en la hoja tu nombre completo, así como una identificación de cada ejercicio.

\begin{enumerate}
\item ¿A qué velocidad promedio iba un auto que recorrió \SI{250}{\kilo\meter} en \SI{3}{\hour}?
\item Calcula el tiempo en minutos de un nadador que batió el récord mundial de los \SI{400}{\meter} libres a una velocidad de \SI{20}{\kilo\meter}.
\item ¿A qué velocidad en \unit{\kilo\meter\per\hour} corrió Usain Bolt en el Campeonato Mundial de Berlín en el año 2009 para batir el récord mundial de los \SI{100}{\meter} planos en \SI{9.58}{\second}?
\item ¿Qué distancia recorrió un avión que viajaba a \SI{750}{\kilo\meter\per\hour} después de \SI{2.5}{\hour} de vuelo?
\item Si en una carretera de Estados Unidos la velocidad máxima es de $80$ mi/\unit{\hour}, ¿cuál es su velocidad en \unit{\kilo\meter\per\hour}?
\item Calcula la velocidad en \unit{\kilo\meter\per\hour}, a la que corrió el atleta keniata Wilson Kipsang Kiprotich para batir el récord mundial vigente, que realizó en el maratón de Berlín, en el año 2013, cuya distancia total es de \SI{42.195}{\kilo\meter}, en un tiempo de \SI{2}{\hour} \SI{3}{\minute} \SI{23}{\second}.
\end{enumerate}

\end{document}