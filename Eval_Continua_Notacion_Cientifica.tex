\documentclass[14pt]{extarticle}
\usepackage[utf8]{inputenc}
\usepackage[T1]{fontenc}
\usepackage[spanish,es-lcroman]{babel}
\usepackage{amsmath}
\usepackage{amsthm}
\usepackage{physics}
\usepackage{tikz}
\usepackage{float}
\usepackage[autostyle,spanish=mexican]{csquotes}
\usepackage[per-mode=symbol]{siunitx}
\usepackage{gensymb}
\usepackage{multicol}
\usepackage{enumitem}
\usepackage[left=2.00cm, right=2.00cm, top=2.00cm, 
     bottom=2.00cm]{geometry}

%\renewcommand{\questionlabel}{\thequestion)}
\decimalpoint
\sisetup{bracket-numbers = false}

\title{\vspace*{-2cm} Ejercicios de Notación Científica \\  Evaluación Continua - Curso de Física 1\vspace{-5ex}}
\date{\today}

\begin{document}
\maketitle

\textbf{Indicaciones:} Resuelve de manera detallada cada uno de los siguientes ejercicios, en donde deberás de indicar el paso (o pasos necesarios) para llegar al resultado.

Cada ejercicio bien resuelto aporta $0.1$ puntos. El total de puntos para esta actividad es de \textbf{2 puntos}.

Si se reporta el resultado directo, cada ejercicio se calificará con la mitad del puntaje que se indica.

\begin{enumerate}
\item \textbf{(0.4) puntos.} Escribe las siguientes cifras en notación científica:
\begin{enumerate}
\item $16200000000000$
\item $0.00000045$
\item $0.00000000123$
\item $384500000000000$
\end{enumerate}
\item \textbf{(0.1) punto.} La velocidad de la luz en el vacío es de \SI{300000}{\kilo\meter\per\second}. Escribe en notación científica esta cifra pero expresada en pulgadas por segundo.
\item \textbf{(0.5) puntos.} Realiza las siguientes operaciones con números en notación científica:
\begin{enumerate}
    \item (\num{3.4d23})(\num{1.1d11})
    \item (\num{2.45d-12})(\num{2.45d-4})
    \item (\num{4.56d11})(\num{0.655d-20})
    \item \num{2.2d5} / \num{0.6d45}
    \item \num{0.68d-8} / \num{1.9d-13}
\end{enumerate}
\item \textbf{(0.4) puntos.} Realiza las siguientes operaciones con números en notación científica:
\begin{enumerate}
    \item \num{3.33d4} + \num{4.5d3}
    \item \num{2.36d-2} + \num{3.68d-5}
    \item \num{4.44d-7} + \num{3.8d6}
    \item \num{1.3d8} + \num{2.5d10}
\end{enumerate}
\item \textbf{(0.6) puntos.} Realiza las siguientes operaciones con números en notación científica:
\begin{enumerate}
    \itemsep1em
    \item $\dfrac{(\num{3.12d-5} + \num{7.03d-4})(\num{8.3d8})}{\num{4.32d3}}$
    \item $\dfrac{\num{5.431d3} - \num{6.51d4} + \num{385.10d2}}{\num{8.2d-3} - \num{2.10d-4}}$
    \item $\dfrac{(\num{12.5d7} - \num{8.10d9})(\num{3.5d-5} + 185)}{\num{9.2d6}}$
    \item $\dfrac{(\num{3d6})(\num{8d-2})}{(\num{2d7}) (a)} = \num{2d-18}$ \quad Obtén el valor de $a$.
    \item El diámetro de un virus es de \num{5d-4}. ¿Cuántos de esos virus son necesarios para rodear a la Tierra?. Considera que el radio medio de la Tierra es de \SI{6370}{\kilo\meter}.
    \item La velocidad de la luz es \SI{3d8}{\meter\per\second}. ¿Qué distancia en \unit{\kilo\meter} recorre la luz en un año?. La expresión que relaciona la velociad ($v$) con la distancia $d$ y el tiempo $t$, es:
    \begin{align*}
    v = \dfrac{d}{t}
    \end{align*}
\end{enumerate}
\end{enumerate}

\end{document}