\documentclass[14pt]{extarticle}
\usepackage[utf8]{inputenc}
\usepackage[T1]{fontenc}
\usepackage[spanish,es-lcroman]{babel}
\usepackage{amsmath}
\usepackage{amsthm}
\usepackage{physics}
\usepackage{tikz}
\usepackage{float}
\usepackage[autostyle,spanish=mexican]{csquotes}
\usepackage[per-mode=symbol]{siunitx}
\usepackage{gensymb}
\usepackage{multicol}
\usepackage{enumitem}
\usepackage[left=2.00cm, right=2.00cm, top=2.00cm, 
     bottom=2.00cm]{geometry}
\usepackage{Estilos/ColoresLatex}

\newcommand{\textocolor}[2]{\textbf{\textcolor{#1}{#2}}}

%\renewcommand{\questionlabel}{\thequestion)}
\decimalpoint
\sisetup{bracket-numbers = false}

\title{\vspace*{-2cm} Ejercicios de Dinámica (Parte 1) - Física 1\vspace{-5ex}}
\date{\today}

\begin{document}
\maketitle

\section{Ejercicios a cuenta}

Estos ejercicios otorgarán hasta $10$ puntos a la Evaluación Continua.

La entrega se hará vía Teams en asignación, teniendo como plazo el día domingo 6 de agosto a las 8 pm.

Cada ejercicio vale $1$ punto, siempre y cuando esté correcto. Se otorgará una parte proporcional en caso de tener desarrollo detallado pero el resultado no sea el esperado.

La solución al ejercicio deberá de desarrollarse de manera completa: datos, expresión, sustitución, manejo de unidades y resultado. En caso de presentar solo el resultado, es decir, sin desarrollo, aunque el resultado sea el correcto, la solución no aportará puntaje.

Anota en la hoja tu nombre completo y todo el enunciado de cada ejercicio.

\begin{enumerate}
\item Determina la magnitud de la fuerza que se debe aplicar a una motoneta que tiene una masa de \SI{40}{\kilogram} para que cambie la magnitud de su velocidad de $0$ a \SI{3}{\meter\per\second} en un segundo.
\item Calcula la masa de un bloque de madera en kilogramos si al recibir una fuerza cuya magnitud es de \SI{300}{\newton} le produce una aceleración con una magnitud de \SI{150}{\centi\meter\per\square\second}.
\item Determina la magnitud de la aceleración en \unit{\meter\per\square\second} que le produce una fuerza cuya magnitud es de \SI{75}{\newton} a una piedra con una masa de \SI{1500}{\gram}.
\item Calcula la magnitud de la fuerza que se le aplica a un sillón de \SI{10}{\kilo\gram} de masa si  adquiere una aceleración con una magnitud de \SI{2.5}{\meter\per\square\second}.
\item Hallar la magnitud del peso de una roca cuya masa es de \SI{100}{\kilo\gram}.
\item Determina la masa de una roca cuyo peso tiene una magnitud de \SI{1500}{\newton}.
\item Calcula la magnitud de la fuerza neta que debe aplicarse a un bloque cuyo peso tiene una magnitud de \SI{25}{\newton} para que adquiera una aceleración cuya magnitud es de \SI{3}{\meter\per\square\second}.
\item Un instante antes de que un prisma rectangular de madera que tiene un peso con una magnitud de \SI{20}{\newton} comience a deslizarse sobre una superficie horizontal de cemento, se produce una fuerza máxima de fricción estática cuya magnitud es de \SI{11}{\newton}. Calcula el coeficiente de fricción estática entre la madera y el cemento.
\item Calcular la magnitud de la fuerza que se debe aplicar para deslizar al bloque de la siguiente figura a velocidad constante, si tiene un peso cuya magnitud es de \SI{150}{N} y el coeficiente de fricción cinética es
de \num{0.3}.
\begin{figure}[H]
    \centering
    \begin{tikzpicture}
        \draw (0, 0) -- (4, 0);
        \draw (1, 0) rectangle (3, 1);
        \draw [dashed] (3, 1) -- (5, 1);
        \draw [-stealth, thick] (3, 1) -- (4.87, 1.68);
        \draw (4, 1) arc(0:20:1);
        \node at (4.8, 1.2) {\small{$\theta = \ang{20}$}};
        \node at (5, 2.3) {\small{$F = \, ?$}};
    \end{tikzpicture}
\end{figure}
\item Una motocicleta cuyo peso tiene una magnitud de \SI{1800}{\newton} se mueve a una velocidad cuya magnitud es de \SI{60}{\kilo\meter\per\hour}. Al aplicar los frenos se detiene a una distancia de \SI{25}{\meter}. Calcula la magnitud de la fuerza de fricción promedio que la detiene.
\end{enumerate}

\end{document}