\documentclass[14pt]{extarticle}
\usepackage[utf8]{inputenc}
\usepackage[T1]{fontenc}
\usepackage[spanish,es-lcroman]{babel}
\usepackage{amsmath}
\usepackage{amsthm}
\usepackage{physics}
\usepackage{tikz}
\usepackage{float}
\usepackage[autostyle,spanish=mexican]{csquotes}
\usepackage[per-mode=symbol]{siunitx}
\usepackage{gensymb}
\usepackage{multicol}
\usepackage{enumitem}
\usepackage[left=2.00cm, right=2.00cm, top=2.00cm, 
     bottom=2.00cm]{geometry}
\usepackage{Estilos/ColoresLatex}
\usepackage{makecell}

\newcommand{\textocolor}[2]{\textbf{\textcolor{#1}{#2}}}
\DeclareSIUnit[number-unit-product = {\,}]\cal{cal}

%\renewcommand{\questionlabel}{\thequestion)}
\decimalpoint
\sisetup{bracket-numbers = false}

\title{\vspace*{-2cm} Práctica 7 - Fricción estática\vspace{-5ex}}
\date{\today}

\begin{document}
\maketitle

\textbf{Nombre:} \rule{8cm}{0.3pt} \hspace{0.5cm} \textbf{Fecha:} \rule{3cm}{0.3pt}

\vspace*{1em}

\textbf{Objetivo: } Obtener de manera experimental el valor del coeficiente de fricción estática y cinemática de varios materiales: madera, unicel, acero inoxidable, moneda de 10 pesos.

\vspace*{1em}

Actividades a realizar.
\begin{enumerate}
\item Obtén el peso de cada uno de los bloques.
\item Coloca un bloque seleccionado sobre el plano horizontal.
\item Registra en la tabla 1 el ángulo del plano y calcula el valor de $\mu_{s}$ con la expresión: $\mu_{s} = \tan \, \theta$
\item Aumenta el ángulo de inclinación gradualmente hasta que inicie el deslizamiento, en cada incremento de inclinación del plano registra el ángulo y calcula $\mu_{s}$.
\item Repite en tres ocasiones el procedimiento para revisar que el ángulo en el que comienza a moverse el bloque, es el mismo.

\textbf{Bloque 1:} \rule{3cm}{0.3pt} 
\\
\textbf{Peso:} \rule{3cm}{0.3pt}

\begin{minipage}{0.45\linewidth}
\begin{table}[H]
    \centering
    \begin{tabular}{| p{1.5cm} | p{1.5cm} | } \hline
        \makecell{$\theta$} & \makecell{$\mu_{s}$} \\ \hline
         &  \\ \hline
         &  \\ \hline
         &  \\ \hline
         &  \\ \hline
         &  \\ \hline
         &  \\ \hline
         &  \\ \hline
    \end{tabular}
\end{table}
\end{minipage}
\begin{minipage}{0.5\linewidth}
\textbf{Repeticiones:}
\begin{table}[H]
    \centering
    \begin{tabular}{| c | p{1.5cm} | p{1.5cm} | } \hline
        Repetición & \makecell{$\theta$} & \makecell{$\mu_{s}$} \\ \hline
        1 &  & \\ \hline
        2 &  & \\ \hline
        3 &  & \\ \hline
    \end{tabular}
\end{table}
\end{minipage}
\item Una vez concluido el procedimiento, repite el cada paso con los diferentes bloques, crea las correspondientes tablas.
\newpage
\item Completa la siguiente tabla:
\begin{table}[H]
    \centering
    \begin{tabular}{|c | p{2cm} |} \hline
        Material & \makecell{$\mu_s$} \\ \hline
        Madera & \\ \hline
        Unicel & \\ \hline
        Acero & \\ \hline
        Moneda & \\ \hline        
    \end{tabular}
\end{table}
\item Esta hoja se deberá de entregar al concluir la clase, la calificación de la práctica será:
\begin{table}[H]
    \centering
    \begin{tabular}{| c | c |} \hline
        Coeficientes & Calificación \\ \hline
        $4$ & $10$ \\ \hline
        $3$ & $8$ \\ \hline
        $2$ & $7$ \\ \hline
        $1$ & $5$ \\ \hline        
    \end{tabular}
\end{table}
\end{enumerate}


\end{document}