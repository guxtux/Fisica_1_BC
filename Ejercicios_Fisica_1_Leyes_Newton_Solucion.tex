\documentclass[14pt]{extarticle}
\usepackage[utf8]{inputenc}
\usepackage[T1]{fontenc}
\usepackage[spanish,es-lcroman]{babel}
\usepackage{amsmath}
\usepackage{amsthm}
\usepackage{physics}
\usepackage{tikz}
\usepackage{float}
\usepackage[autostyle,spanish=mexican]{csquotes}
\usepackage[per-mode=symbol]{siunitx}
\usepackage{gensymb}
\usepackage{multicol}
\usepackage{enumitem}
\usepackage[left=2.00cm, right=2.00cm, top=2.00cm, 
     bottom=2.00cm]{geometry}
\usepackage{Estilos/ColoresLatex}
\usepackage{makecell}

\newcommand{\textocolor}[2]{\textbf{\textcolor{#1}{#2}}}
\DeclareSIUnit[number-unit-product = {\,}]\cal{cal}

%\renewcommand{\questionlabel}{\thequestion)}
\decimalpoint
\sisetup{bracket-numbers = false}

\title{\vspace*{-2cm} Ejercicios Leyes de Newton - Solución\vspace{-5ex}}
\date{\today}

\begin{document}
\maketitle

Esta actividad otorgará hasta \textbf{8 puntos} de Evaluación Continua.
\vspace*{0.5cm}

\textbf{Instrucciones: }
\begin{itemize}
\item Anota tu nombre en cada hoja que ocupes para resolver los ejercicios.
\item Identifica el ejercicio que resuelves.
\item Resuelve detalladamente cada ejercicio, en caso de que no se tenga el desarrollo, no se tomará en cuenta como ejercicio completo.
\item Revisa y detalla con cuidado el manejo de las unidades.
\item En caso de plagios, se cancelarán todos los trabajos involucrados.
\end{itemize}

\section{Ejercicios a cuenta.}

\begin{enumerate}
\item Ordena las siguientes situaciones de acuerdo con la magnitud de la aceleración del objeto, de la más baja a la más alta.
\begin{enumerate}[label=\alph*)]
\item Sobre un objeto de \SI{2.0}{\kilo\gram} actúa una fuerza neta de \SI{2.0}{\newton}
\item Sobre un objeto de \SI{2.0}{\kilo\gram} actúa una fuerza neta de \SI{8.0}{\newton}
\item Sobre un objeto de \SI{8.0}{\kilo\gram} actúa una fuerza neta de \SI{2.0}{\newton}
\item Sobre un objeto de \SI{8.0}{\kilo\gram} actúa una fuerza neta de \SI{8.0}{\newton}
\end{enumerate}
a) $a = \dfrac{\SI{2.0}{\newton}}{\SI{2.0}{\kilo\gram}} = \SI{1}{\meter\per\square\second}$

b) $a = \dfrac{\SI{8.0}{\newton}}{\SI{2.0}{\kilo\gram}} = \SI{4}{\meter\per\square\second}$

c) $a = \dfrac{\SI{2.0}{\newton}}{\SI{8.0}{\kilo\gram}} = \SI{0.25}{\meter\per\square\second}$

d) $a = \dfrac{\SI{8.0}{\newton}}{\SI{8.0}{\kilo\gram}} = \SI{1}{\meter\per\square\second}$

Por lo tanto, de menor a mayor aceleración se tiene:

c) \quad a) \quad c) \quad b)
\item Determina la fuerza que se necesita aplicar a un coche de \SI{800}{\kilo\gram} para que éste se acelere \SI{4}{\meter\per\square\second}.
\begin{align*}
F = m \, a = (\SI{800}{\kilo\gram})(\SI{4}{\meter\per\square\second}) = \SI{3200}{\newton}
\end{align*}
\item El resultado de las fuerzas que actúan sobre un cuerpo cuya masa vale \SI{40}{\kilo\gram}, es de \SI{85}{\newton}. ¿Cuál es el valor de la aceleración que experimenta este objeto?
\begin{align*}
a = \dfrac{F}{m} = \dfrac{\SI{85}{\newton}}{\SI{40}{\kilo\gram}} = \SI{2.12}{\meter\per\square\second}
\end{align*}    
\item ¿Cuál es la masa de un cuerpo si al aplicarle una fuerza de \SI{420}{\newton} adquiere una aceleración de \SI{8.4}{\meter\per\square\second}?
\begin{align*}
m = \dfrac{F}{a} = \dfrac{\SI{420}{\newton}}{\SI{8.4}{\meter\per\square\second}} = \SI{50}{\kilogram}
\end{align*}
\item Calcula la magnitud de la aceleración que produce una fuerza cuya magnitud es de \SI{100}{\newton} a un triciclo cuya masa es de \SI{6000}{\gram}. Expresa el resultado en \unit{\meter\per\square\second}.
\begin{align*}
a = \dfrac{F}{m} = \dfrac{\SI{100}{\newton}}{\SI{6}{\kilo\gram}} = \SI{16.66}{\meter\per\square\second}
\end{align*}
\item Determina la magnitud de la fuerza neta que debe aplicarse a un balón cuyo peso tiene una magnitud de
\SI{4.2}{\newton} para que adquiera una aceleración cuya magnitud es de \SI{3}{\meter\per\square\second}.
\begin{align*}
F = m \, a = (\dfrac{\SI{4.2}{\kilo\gram\meter\per\square\second}}{\SI{9.81}{\meter\per\square\second}} )(\SI{3}{\meter\per\square\second}) = \SI{1.28}{\newton}
\end{align*}    
\item Calcula la masa de una caja en kilogramos, si al recibir una fuerza cuya magnitud es de \SI{300}{\newton} le produce una aceleración con una magnitud de \SI{150}{\centi\meter\per\square\second}.
\begin{align*}
m = \dfrac{F}{a} = \dfrac{\SI{300}{\newton}}{\SI{1.5}{\meter\per\square\second}} = \SI{200}{\kilogram}
\end{align*}    
\item Determina la magnitud de la aceleración en \unit{\meter\per\square\second} que le produce una fuerza cuya magnitud de \SI{75}{\newton} a una silla cuya masa es de \SI{1500}{\gram}.
\begin{align*}
a = \dfrac{F}{m} = \dfrac{\SI{75}{\newton}}{\SI{1.5}{\kilo\gram}} = \SI{50}{\meter\per\square\second}
\end{align*}    
\end{enumerate}


\end{document}