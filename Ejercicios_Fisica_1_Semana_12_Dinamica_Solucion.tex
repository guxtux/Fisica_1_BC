\documentclass[14pt]{extarticle}
\usepackage[utf8]{inputenc}
\usepackage[T1]{fontenc}
\usepackage[spanish,es-lcroman]{babel}
\usepackage{amsmath}
\usepackage{amsthm}
\usepackage{physics}
\usepackage{tikz}
\usepackage{float}
\usepackage[autostyle,spanish=mexican]{csquotes}
\usepackage[per-mode=symbol]{siunitx}
\usepackage{gensymb}
\usepackage{multicol}
\usepackage{enumitem}
\usepackage[left=2.00cm, right=2.00cm, top=2.00cm, 
     bottom=2.00cm]{geometry}
\usepackage{Estilos/ColoresLatex}

\newcommand{\textocolor}[2]{\textbf{\textcolor{#1}{#2}}}

%\renewcommand{\questionlabel}{\thequestion)}
\decimalpoint
\sisetup{bracket-numbers = false}

\title{\vspace*{-2cm} Ejercicios de Dinámica (Parte 1) Solución\vspace{-5ex}}
\date{\today}

\begin{document}
\maketitle

\section{Ejercicios a cuenta}

\begin{enumerate}
\item Determina la magnitud de la fuerza que se debe aplicar a una motoneta que tiene una masa de \SI{40}{\kilogram} para que cambie la magnitud de su velocidad de $0$ a \SI{3}{\meter\per\second} en un segundo.
\begin{align*}
a &= \dfrac{v_{f} - v_{i}}{t} = \dfrac{\SI{3}{\meter\per\second} - 0}{\SI{1}{\second}} = \SI{3}{\meter\per\square\second} \\[0.5em]
F &= m \, a = (\SI{40}{\kilo\gram})(\SI{3}{\meter\per\square\second}) = \SI{120}{\newton}
\end{align*}
\item Calcula la masa de un bloque de madera en kilogramos si al recibir una fuerza cuya magnitud es de \SI{300}{\newton} le produce una aceleración con una magnitud de \SI{150}{\centi\meter\per\square\second}.
\begin{align*}
m = \dfrac{F}{a} = \dfrac{\SI{300}{\newton}}{\SI{1.5}{\meter\per\square\second}} = \SI{200}{\kilo\gram}
\end{align*}
\item Determina la magnitud de la aceleración en \unit{\meter\per\square\second} que le produce una fuerza cuya magnitud es de \SI{75}{\newton} a una piedra con una masa de \SI{1500}{\gram}.
\begin{align*}
a = \dfrac{F}{m} = \dfrac{\SI{75}{\newton}}{\SI{1.5}{\kilo\gram}} = \SI{50}{\meter\per\square\second}
\end{align*}
\item Calcula la magnitud de la fuerza que se le aplica a un sillón de \SI{10}{\kilo\gram} de masa si  adquiere una aceleración con una magnitud de \SI{2.5}{\meter\per\square\second}.
\begin{align*}
F = m \, a = (\SI{10}{\kilo\gram})(\SI{2.5}{\meter\per\square\second}) = \SI{25}{\newton}
\end{align*}
\item Hallar la magnitud del peso de una roca cuya masa es de \SI{100}{\kilo\gram}.
\begin{align*}
W = m \, g = (\SI{100}{\kilo\gram})(\SI{9.81}{\meter\per\square\second}) = \SI{981}{\newton}
\end{align*}
\item Determina la masa de una roca cuyo peso tiene una magnitud de \SI{1500}{\newton}.
\begin{align*}
m = \dfrac{W}{g} = \dfrac{\SI{1500}{\newton}}{\SI{9.81}{\meter\per\square\second}} = \SI{152.90}{\kilo\gram}
\end{align*}
\item Calcula la magnitud de la fuerza neta que debe aplicarse a un bloque cuyo peso tiene una magnitud de \SI{25}{\newton} para que adquiera una aceleración cuya magnitud es de \SI{3}{\meter\per\square\second}.
\begin{align*}
F = m \, a = \left( \dfrac{\SI{25}{\newton}}{\SI{9.81}{\meter\per\square\second}} \right) (\SI{3}{\meter\per\square\second}) = (\SI{2.54}{\kilo\gram})(\SI{3}{\meter\per\square\second}) = \SI{7.62}{\newton}
\end{align*}
\item Un instante antes de que un prisma rectangular de madera que tiene un peso con una magnitud de \SI{20}{\newton} comience a deslizarse sobre una superficie horizontal de cemento, se produce una fuerza máxima de fricción estática cuya magnitud es de \SI{11}{\newton}. Calcula el coeficiente de fricción estática entre la madera y el cemento.
\begin{align*}
f_{s} &= \mu_{s} \, N \hspace{0.3cm} \Rightarrow \hspace{0.3cm} \mu_{s} = \dfrac{f_{s}}{N} \\[0.5em]
\mu_{s} &= \dfrac{\SI{11}{\newton}}{\SI{20}{\newton}} = \num{0.55}
\end{align*}
\item Calcular la magnitud de la fuerza que se debe aplicar para deslizar al bloque de la siguiente figura a velocidad constante, si tiene un peso cuya magnitud es de \SI{150}{N} y el coeficiente de fricción cinética es
de \num{0.3}.
\begin{figure}[H]
    \centering
    \begin{tikzpicture}
        \draw (0, 0) -- (4, 0);
        \draw (1, 0) rectangle (3, 1);
        \draw [dashed] (3, 1) -- (5, 1);
        \draw [-stealth, thick] (3, 1) -- (4.87, 1.68);
        \draw (4, 1) arc(0:20:1);
        \node at (4.8, 1.2) {\small{$\theta = \ang{20}$}};
        \node at (5, 2.3) {\small{$F = \, ?$}};
    \end{tikzpicture}
\end{figure}
Hacemos el diagrama de cuerpo libre:
\begin{figure}[H]
    \centering
    \begin{tikzpicture}[scale=1.5]
        \draw (-2, 0) -- (2.5, 0);
        \draw (0, -2.5) -- (0, 2.5);
        \draw [-stealth, thick] (0, 0) -- (1.87, 0.68);
        \draw (0.5, 0) arc(0:20:0.5);
        \node at (1.2, 0.2) {\small{$\theta$}};
        % \node at (5, 2.3) {\small{$F = \, ?$}};
        \draw [-stealth, thick] (0, 0) -- (1.87, 0) node [below, midway] {\small{$F_{x}$}};
        \draw [-stealth, thick] (1.87, 0) -- (1.87, 0.68) node [right, midway] {\small{$F_{y}$}};
        \draw [-stealth, thick] (0, 0) -- (-1.5, 0) node [above, pos=1] {\small{$f_{k}$}};
        \draw [-stealth, thick] (0, 0) -- (0, 2) node [left, pos=1] {\small{$N$}};
        \draw [-stealth, thick] (0, 0) -- (0, -2) node [left, pos=1] {\small{$W$}};
    \end{tikzpicture}
\end{figure}
Se tiene entonces que:
\begin{align}
\sum F_{x} &= F_{x} - f_{k} = 0 \label{eq:ecuacion_01} \\[0.5em]
\sum F_{y} &= N + (-W) + F_{y} = 0 \label{eq:ecuacion_02}
\end{align}
De la ecuación (\ref{eq:ecuacion_01}):
\begin{align}
F_{x} = f_{k} = \mu_{k} \, N \label{eq:ecuacion_03}
\end{align}
De la ecuación (\ref{eq:ecuacion_02}):
\begin{align}
N = P - F_{y} \label{eq:ecuacion_04}
\end{align}
Sustituyendo la ec. (\ref{eq:ecuacion_04}) en la ec. (\ref{eq:ecuacion_03}):
\begin{align}
F_{x} = \mu_{k} (P - F_{y}) \label{eq:ecuacion_05}
\end{align}
y como:
\begin{align}
F_{x} &= F \, \cos \ang{20} = F \, (0.9397) \label{eq:ecuacion_06} \\[0.5em]
F_{y} &= F \, \sin \ang{20} = F \, (0.3420) \label{eq:ecuacion_07}
\end{align}
Sustituyendo las ecs. (\ref{eq:ecuacion_06}) y (\ref{eq:ecuacion_07}) en la ec. (\ref{eq:ecuacion_05}):
\begin{align*}
F \, (0.9397) &= 0.3 ( \SI{150}{\newton} - F \, (0.3420)) \\[0.5em]
F \, (0.9397) &= \SI{45}{\newton} - F \, (0.1) \\[0.5em]
F \, (0.9397) + F \, (0.1) &= \SI{45}{\newton} \\[0.5em]
F &= \dfrac{\SI{45}{\newton}}{1.0397} = \SI{43.28}{\newton}
\end{align*}
\item Una motocicleta cuyo peso tiene una magnitud de \SI{1800}{\newton} se mueve a una velocidad cuya magnitud es de \SI{60}{\kilo\meter\per\hour}. Al aplicar los frenos se detiene a una distancia de \SI{25}{\meter}. Calcula la magnitud de la fuerza de fricción promedio que la detiene.
\begin{align*}
v_{i} &= \SI[per-mode=fraction]{60}{\kilo\meter\per\hour}\left( \dfrac{\SI{1000}{\meter}}{\SI{1}{\kilo\meter}} \right) \left( \dfrac{\SI{1}{\hour}}{\SI{3600}{\second}} \right) = \SI{16.66}{\meter\per\second} \\[0.5em]
v_{f}^{2} &= v_{i}^{2} + 2 \, a \, d \hspace{0.3cm} \Rightarrow \hspace{0.3cm} 0 = v_{0}^{2} + 2 \, a \, d \\[0.5em]
a &= - \dfrac{v_{0}^{2}}{2 \, d} = - \dfrac{(\SI{16.66}{\meter\per\second})^{2}}{2 (\SI{25}{\meter})} = - \SI{5.55}{\meter\per\square\second} \\[0.5em]
F &= \dfrac{W}{g} \, a = \dfrac{\SI{1800}{\newton}}{\SI{9.81}{\meter\per\square\second}} (- \SI{5.55}{\meter\per\square\second}) = - \SI{1019.39}{\newton}
\end{align*}
\end{enumerate}

\end{document}