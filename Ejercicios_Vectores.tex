\documentclass[14pt]{extarticle}
\usepackage[utf8]{inputenc}
\usepackage[T1]{fontenc}
\usepackage[spanish,es-lcroman]{babel}
\usepackage{amsmath}
\usepackage{amsthm}
\usepackage{physics}
\usepackage{tikz}
\usepackage{float}
\usepackage[autostyle,spanish=mexican]{csquotes}
\usepackage[per-mode=symbol]{siunitx}
\usepackage{gensymb}
\usepackage{multicol}
\usepackage{enumitem}
\usepackage[left=2.00cm, right=2.00cm, top=2.00cm, 
     bottom=2.00cm]{geometry}
\usepackage{Estilos/ColoresLatex}

%\renewcommand{\questionlabel}{\thequestion)}
\decimalpoint
\sisetup{bracket-numbers = false}

\title{\vspace*{-2cm} Ejercicios de Vectores \\  Curso de Física 1\vspace{-5ex}}
\date{\today}

\begin{document}
\maketitle

\section{Ejercicios a cuenta}

\subsection{Ejercicio 1}

Realiza la suma de los siguientes tres vectores, determinando el vector resultante y el ángulo que forma la resultante con respecto al eje $x$ positivo.

\begin{figure}[H]
\centering
\begin{tikzpicture}[scale=1.3]
    \draw (-6, 0) -- (2, 0);
    \draw (0, -2) -- (0, 1.5);
    \draw [-stealth, line width=0.5mm, color=ao] (0, 0) -- (1.69, 0.61) node [above, pos=1.2] {\small{$T_{1} = \SI{18}{\newton}$}};
    \draw [color=ao] (0.5, 0) arc(0:20:0.4);
    \node at (2, 0.2) [color=ao] {\small{$\theta_{1} = \ang{20}$}};

    \draw [-stealth, line width=0.5mm, color=darkmagenta] (0, 0) -- (-5.6, 0) node [above, near end] {\small{$T_{3} = \SI{58}{\newton}$}};
    \draw [color=darkmagenta] (0.4, 0) arc(0:180:0.5);
    \node at (-1.5, 0.4) [color=darkmagenta] {\small{$\theta_{3} = \ang{180}$}};
    
    \draw [-stealth, line width=0.5mm, color=officegreen] (0, 0) -- (0.306, -0.629) node [above, xshift=0.5cm, yshift=-0.7cm] {\small{$T_{2} = \SI{7}{\newton}$}};
    \draw [color=officegreen] (0.3, 0) arc(360:296:0.3);
    \node at (1.3, -0.4) [color=officegreen] {\small{$\theta_{2} = \ang{64}$}};
\end{tikzpicture}
\end{figure}

\newpage

\subsection{Ejercicio 2}

Realiza la suma de los vectores que se muestran en la siguiente figura, calculando la magnitud del vector resultante, así como la dirección del mismo.

\begin{figure}[H]
    \centering
    \begin{tikzpicture}[scale=1]
    \draw (-4, 0) -- (5, 0);
    \draw (0, -9) -- (0, 5);
    \draw [-stealth, line width=0.5mm, color=carmine] (0, 0) -- (4.17, 4.17) node [above, pos=1] {\small{$F_{1} = \SI{59}{\newton}$}};
    \draw [color=carmine] (0.5, 0) arc(0:45:0.5);
    \node at (2.5, 0.3) [color=carmine] {\small{$\theta_{1} = \ang{45}$}};

    \draw [-stealth, line width=0.5mm, color=electricindigo] (0, 0) -- (1.97, -1.18) node [above, xshift=0.8cm, yshift=-0.7cm] {\small{$F_{2} = \SI{23}{\newton}$}};
    \draw [color=electricindigo] (0.5, 0) arc(360:329:0.5);
    \node at (2.8, -0.5) [color=electricindigo] {\small{$\theta_{2} = \ang{31}$}};
    
    \draw [-stealth, line width=0.5mm, color=officegreen] (0, 0) -- (-0.919, -8.75) node [left, midway] {\small{$F_{3} = \SI{88}{\newton}$}};
    \draw [color=officegreen] (-0.5, 0) arc(180:264:0.5);
    \node at (-2, -0.5) [color=officegreen] {\small{$\theta_{3} = \ang{84}$}};

    \draw [-stealth, line width=0.5mm, color=persimmon] (0, 0) -- (-0.745, 1.52) node [above, xshift=-1cm, yshift=-0.2cm] {\small{$F_{4} = \SI{17}{\newton}$}};
    \draw [thick, color=persimmon] (-0.5, 0) arc(180:116:0.5);
    \node at (-2, 0.5) [color=persimmon] {\small{$\theta_{4} = \ang{64}$}};
\end{tikzpicture}
\end{figure}

Cada ejercicio aporta un punto en Evaluación Continua. La solución de enviará por Teams, el plazo vence el domingo 2 de julio a las 8 pm.


No se recibirán trabajos extemporáneos, a menos que se tenga la evidencia de problemas durante el envío, y que hayan sido notificados al momento, tanto en la Coordinación Académica como al Profesor.
\end{document}