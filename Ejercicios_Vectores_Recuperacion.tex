\documentclass[14pt]{extarticle}
\usepackage[utf8]{inputenc}
\usepackage[T1]{fontenc}
\usepackage[spanish,es-lcroman]{babel}
\usepackage{amsmath}
\usepackage{amsthm}
\usepackage{physics}
\usepackage{tikz}
\usepackage{float}
\usepackage[autostyle,spanish=mexican]{csquotes}
\usepackage[per-mode=symbol]{siunitx}
\usepackage{gensymb}
\usepackage{multicol}
\usepackage{enumitem}
\usepackage[left=2.00cm, right=2.00cm, top=2.00cm, 
     bottom=2.00cm]{geometry}
\usepackage{Estilos/ColoresLatex}

%\renewcommand{\questionlabel}{\thequestion)}
\decimalpoint
\sisetup{bracket-numbers = false}

\title{\vspace*{-2cm} Ejercicios de Vectores \\  Curso de Física 1\vspace{-5ex}}
\date{}

\begin{document}
\maketitle

\section{Ejercicio a cuenta}

% \subsection{Ejercicio 1}

Resolviendo debidamente el ejercicio, se te otorgará $1$ punto en Evaluación Continua.


Realiza la suma de los siguientes tres vectores, determinando el vector resultante y el ángulo que forma la resultante con respecto al eje $x$ positivo.

\begin{figure}[H]
\centering
\begin{tikzpicture}[scale=1.3]
    \draw (-6, 0) -- (4, 0) node [above, pos=1] {$x$} ;
    \draw (0, -2) -- (0, 4) node [left, pos=1] {$y$};
    \draw [-stealth, line width=0.5mm, color=ao] (0, 0) -- (-5.46, 2.53) node [above, xshift=0.5cm, yshift=0.1cm] {\small{$F_{1} = \SI{15}{\newton}$}};
    \draw [color=ao] (-0.7, 0) arc(180:155:0.7);
    \node at (-1.5, 0.2) [color=ao] {\small{$\theta_{1} = \ang{25}$}};

    \draw [-stealth, line width=0.5mm, color=darkmagenta] (0, 0) -- (3.8, -1.23) node [above, xshift=0.5cm, yshift=-0.7cm] {\small{$F_{2} = \SI{10}{\newton}$}};
    \draw [color=darkmagenta] (0.7, 0) arc(360:342:0.7);
    \node at (2, -0.2) [color=darkmagenta] {\small{$\theta_{2} = \ang{18}$}};
    
    \draw [-stealth, line width=0.5mm, color=officegreen] (0, 0) -- (2.82, 3.36) node [above, xshift=0.9cm, yshift=-0.7cm] {\small{$T_{2} = \SI{7}{\newton}$}};
    \draw [color=officegreen] (0.7, 0) arc(0:50:0.7);
    \node at (1.3, 0.5) [color=officegreen] {\small{$\theta_{3} = \ang{50}$}};
\end{tikzpicture}
\end{figure}
\end{document}