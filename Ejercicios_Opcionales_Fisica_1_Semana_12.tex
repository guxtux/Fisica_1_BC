\documentclass[14pt]{extarticle}
\usepackage[utf8]{inputenc}
\usepackage[T1]{fontenc}
\usepackage[spanish,es-lcroman]{babel}
\usepackage{amsmath}
\usepackage{amsthm}
\usepackage{physics}
\usepackage{tikz}
\usepackage{float}
\usepackage[autostyle,spanish=mexican]{csquotes}
\usepackage[per-mode=symbol]{siunitx}
\usepackage{gensymb}
\usepackage{multicol}
\usepackage{enumitem}
\usepackage[left=2.00cm, right=2.00cm, top=2.00cm, 
     bottom=2.00cm]{geometry}
\usepackage{Estilos/ColoresLatex}

\newcommand{\textocolor}[2]{\textbf{\textcolor{#1}{#2}}}

%\renewcommand{\questionlabel}{\thequestion)}
\decimalpoint
\sisetup{bracket-numbers = false}

\title{\vspace*{-2cm} Ejercicios Opcionales - Física 1\vspace{-5ex}}
\date{\today}

\begin{document}
\maketitle

\section{Ejercicios a cuenta.}

Con la finalidad de apoyar en la recuperación del promedio para el tercer examen parcial, se presentan los siguientes \textocolor{red}{ejercicios adicionales} para Evaluación Continua. 


Estos ejercicios serán de carácter \textocolor{cobalt}{opcional}, es decir, la alumna o alumno que desee resolverlos y enviarlos, les sumará $5$ puntos adicionales a la Evaluación Continua. La máxima calificación posible en evaluación continua es $10$. Se recomienda que resuelvan y envíen estos ejercicios, ya que les permitirá asegurar un $10$ en evaluación continua, siempre y cuando hayan enviado los ejercicios semanales.

\textbf{MUY IMPORTANTE:} Para que esta actividad cuente, se deben de resolver y enviar \textbf{TODOS} los ejercicios, de otra manera, no se contabilizarán los ejercicios que se envíen.

La entrega se hará vía Teams en asignación, teniendo como plazo el día viernes 11 de agosto a las 8 pm.

Anota en la hoja tu nombre completo, así como una identificación de cada ejercicio.

\subsection{Fricción.}

\begin{enumerate}
\item Se aplica una fuerza máxima de fricción estática cuya magnitud es de \SI{230}{\newton} sobre una caja de cartón cuyo peso es de \SI{300}{\newton}, un instante antes de que comience a deslizarse sobre una superficie horizontal de granito. Calcula el coeficiente de fricción estático entre el cartón y el granito.
\item Un bloque de madera de \SI{20}{\newton} se jala con una fuerza máxima estática cuya magnitud es de \SI{12}{\newton}; al tratar de deslizarlo sobre una superficie horizontal de madera, ¿cuál es el coeficiente de fricción estático entre las dos superficies?
\item Se aplica una fuerza cuya magnitud es de \SI{85}{\newton} sobre un bloque de madera para deslizarlo a velocidad constante sobre una superficie horizontal. Si la masa del bloque de madera es de \SI{21.7}{\kilo\gram}, ¿cuál es el coeficiente de fricción cinético?
\item Se requiere mover un bloque cuyo peso es de \SI{30}{\newton} sobre una superficie horizontal a una velocidad constante. Si el coeficiente de fricción cinético es de \num{0.5}, determina la magnitud de la uerza que se necesita para moverlo.
\end{enumerate}

\subsection{Fuerza de atracción gravitacional.}
\begin{enumerate}
\item Un muchacho cuya masa es de \SI{60}{\kilo\gram} se encuentra a una distancia de \SI{0.4}{\meter} de una muchacha cuya masa es de \SI{48}{\kilo\gram}, determina la magnitud de la fuerza gravitacional con la cual se atraen.
\item Determina la magnitud de la fuerza gravitacional con la que se atraen un miniauto de \SI{1200}{\kilo\gram} con un camión de carga de \SI{4500}{\kilo\gram}, al estar separados a una distancia de \SI{5}{\meter}.
\item Una barra metálica cuyo peso tiene una magnitud de 800 N se acerca a otra de \SI{1200}{\newton} hasta que la distancia entre sus centros de gravedad es de \SI{80}{\centi\meter}. ¿Con qué magnitud de fuerza se atraen?
\item ¿A qué distancia se encuentran dos elefantes cuyas masas son \SI{1.2d3}{\kilo\gram} y \SI{1.5d3}{\kilo\gram}, y se atraen con una fuerza gravitacional cuya magnitud es de \SI{4.8d3}{\newton}?
\item Determina la masa de una roca, si la fuerza gravitacional con que se atrae con otra de \SI{100}{\kilo\gram} tiene una magnitud de \SI{60d10}{\newton} y la distancia entre ellas es de \SI{10}{\meter}.
\end{enumerate}

\subsection{Trabajo mecánico.}

\begin{enumerate}
\item Se realiza un trabajo mecánico de \SI{3500}{\joule} para elevar una cubeta con arena cuyo peso tiene una magnitud de \SI{350}{\newton}. Determina la altura a la que se elevó la cubeta.
\item Una persona levanta una silla cuyo peso tiene una magnitud de \SI{49}{\newton} hasta una altura de \SI{0.75}{\meter}. ¿Qué trabajo realiza?
\item Determina el trabajo realizado al desplazar un bloque \SI{3}{\meter} sobre una superficie horizontal, si se desprecia la fricción y la fuerza aplicada tiene una magnitud de \SI{25}{\newton}.
\item ¿Qué magnitud de peso tendrá un objeto si al levantarlo a una altura de \SI{1.5}{\newton} se realiza un trabajo de \SI{88.2}{\joule}?
\item Un ladrillo tiene una masa de \SI{1}{\kilo\gram}. ¿A qué distancia se levantó del suelo si se realizó un trabajo de \SI{19.6}{\joule}?
\end{enumerate}

\end{document}