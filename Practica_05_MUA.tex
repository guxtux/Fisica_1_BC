\documentclass[14pt]{beamer}
\usepackage{./Estilos/BeamerUVM}
\usepackage{./Estilos/ColoresLatex}
%Sección para el tema de beamer, con el theme, usercolortheme y sección de footers
\usetheme{CambridgeUS}
\usecolortheme{default}
%\useoutertheme{default}
\setbeamercovered{invisible}
% or whatever (possibly just delete it)
\setbeamertemplate{section in toc}[sections numbered]
\setbeamertemplate{subsection in toc}[subsections numbered]
\setbeamertemplate{subsection in toc}{\leavevmode\leftskip=3.2em\rlap{\hskip-2em\inserttocsectionnumber.\inserttocsubsectionnumber}\inserttocsubsection\par}
\setbeamercolor{section in toc}{fg=blue}
\setbeamercolor{subsection in toc}{fg=blue}
\setbeamercolor{frametitle}{fg=blue}
\setbeamertemplate{caption}[numbered]

\setbeamertemplate{footline}
\beamertemplatenavigationsymbolsempty
\setbeamertemplate{headline}{}


\makeatletter
\setbeamercolor{secºtion in foot}{bg=gray!30, fg=black!90!orange}
\setbeamercolor{subsection in foot}{bg=blue!30!yellow, fg=red}
\setbeamercolor{date in foot}{bg=black, fg=white}
\setbeamertemplate{footline}
{
  \leavevmode%
  \hbox{%
  \begin{beamercolorbox}[wd=.333333\paperwidth,ht=2.25ex,dp=1ex,center]{section in foot}%
    \usebeamerfont{section in foot} \insertsection
  \end{beamercolorbox}%
  \begin{beamercolorbox}[wd=.333333\paperwidth,ht=2.25ex,dp=1ex,center]{subsection in foot}%
    \usebeamerfont{subsection in foot}  \insertsubsection
  \end{beamercolorbox}%
  \begin{beamercolorbox}[wd=.333333\paperwidth,ht=2.25ex,dp=1ex,right]{date in head/foot}%
    \usebeamerfont{date in head/foot} \insertshortdate{} \hspace*{2em}
    \insertframenumber{} / \inserttotalframenumber \hspace*{2ex} 
  \end{beamercolorbox}}%
  \vskip0pt%
}






% \usefonttheme{serif}
\usepackage[clock]{ifsym}
\DeclareSIUnit\erg{erg}
\DeclareSIUnit[number-unit-product = {\,}]\cal{cal}

\sisetup{per-mode=symbol}
\resetcounteronoverlays{saveenumi}

\title{\Large{5 - Mov. Uniformemente Acelerado} \\ \normalsize{Física 2}}
\date{14 de julio de 2023}

\begin{document}
\maketitle

\section*{Contenido}
\frame{\frametitle{Contenido} \tableofcontents[currentsection, hideallsubsections]}

\section{La Práctica}
\frame{\tableofcontents[currentsection, hideothersubsections]}
\subsection{Objetivo}

\begin{frame}
\frametitle{Objetivo de la Práctica}
Determinar la velocidad de un balín deslizándose sobre un plano inclinado.
\end{frame}

\subsection{Material}

\begin{frame}
\frametitle{Material y equipo}
\setbeamercolor{item projected}{bg=bananayellow,fg=blue}
\setbeamertemplate{enumerate items}{%
\usebeamercolor[bg]{item projected}%
\raisebox{1.5pt}{\colorbox{bg}{\color{fg}\footnotesize\insertenumlabel}}%
}
\begin{enumerate}[<+->]
\item Riel.
\item Soporte con transportador.
\item Una canica o balín.
\item Masking tape.
\item Cronómetro (del celular)
\end{enumerate}
\end{frame}

\subsection{Montaje}

\begin{frame}
\frametitle{Montando el riel}
\setbeamercolor{item projected}{bg=burgundy,fg=white}
\setbeamertemplate{enumerate items}{%
\usebeamercolor[bg]{item projected}%
\raisebox{1.5pt}{\colorbox{bg}{\color{fg}\footnotesize\insertenumlabel}}%
}
\begin{enumerate}[<+->]
\item El soporte debe de ajustarse de tal manera que se levanta a un ángulo de $\ang{5}$.
\item Se coloca el riel sobre el soporte.
\item La canica o balín debe de colocarse en la marca inicial del riel.
\seti
\end{enumerate}
\end{frame}
\begin{frame}
\frametitle{Registrando datos}
\setbeamercolor{item projected}{bg=burgundy,fg=white}
\setbeamertemplate{enumerate items}{%
\usebeamercolor[bg]{item projected}%
\raisebox{1.5pt}{\colorbox{bg}{\color{fg}\footnotesize\insertenumlabel}}%
}
    \begin{enumerate}[<+->]
\conti    
\item Se debe de registrar el tiempo que tarda en recorrer las marcas desde \SI{10}{\centi\meter}, \SI{20}{\centi\meter}, \SI{30}{\centi\meter}, hasta \SI{100}{\centi\meter}, es decir, cada hasta \SI{100}{\centi\meter}
\item Se repite $5$ veces el registro de tiempo para cada marca de distancia.
\end{enumerate}
\end{frame}
\begin{frame}
\frametitle{Registrando datos}
\setbeamercolor{item projected}{bg=burgundy,fg=white}
\setbeamertemplate{enumerate items}{%
\usebeamercolor[bg]{item projected}%
\raisebox{1.5pt}{\colorbox{bg}{\color{fg}\footnotesize\insertenumlabel}}%
}
    \begin{enumerate}[<+->]
\conti    
\item Una vez completado el registro de tiempos, se levanta el riel a un ángulo de $\ang{10}$
\item Se repite el procedimiento de registrar $5$ veces el tiempo para cada marca de distancia, de \SI{10}{\centi\meter} a \SI{100}{\centi\meter}
\end{enumerate}
\end{frame}
\begin{frame}
\frametitle{Registrando datos}
\setbeamercolor{item projected}{bg=burgundy,fg=white}
\setbeamertemplate{enumerate items}{%
\usebeamercolor[bg]{item projected}%
\raisebox{1.5pt}{\colorbox{bg}{\color{fg}\footnotesize\insertenumlabel}}%
}
    \begin{enumerate}[<+->]
\conti    
\item Una vez completado el registro de tiempos, se levanta el riel a un ángulo de $\ang{20}$
\item Se repite el procedimiento de registrar $5$ veces el tiempo para cada marca de distancia, de \SI{10}{\centi\meter} a \SI{100}{\centi\meter}
\end{enumerate}
\end{frame}

\section{Analizando los datos}
\frame{\tableofcontents[currentsection, hideothersubsections]}
\subsection{Construyendo tablas}

\begin{frame}
\frametitle{Obteniendo los promedios de tiempo}
Deben de calcular el tiempo promedio a partir de los $5$ registros que obtuvieron para cada marca de distancia.
\end{frame}
\begin{frame}
\frametitle{Primera tabla}
Deberán de completar una tabla con las siguientes variables:
\begin{table}
    \centering
    \begin{tabular}{c | c}
        Tiempo [\unit{\second}] & Velocidad [\unit{\meter}] \\ \hline
         & \\ \hline       
         & \\ \hline       
    \end{tabular}
\end{table}
\end{frame}
\begin{frame}
\frametitle{Graficando los datos}
Ahora vas a graficar los datos de la tabla anterior:
\begin{figure}
    \centering
    \begin{tikzpicture}
        \draw [-stealth] (0, 0) -- (5, 0) node [above, pos=1] {\small{$t \, [\unit{\second}]$}};
        \draw [-stealth] (0, 0) -- (0, 3) node [left, pos=1] {\small{$d \, [\unit{\meter}]$}};
    \end{tikzpicture}
\end{figure}
\end{frame}
\begin{frame}
\frametitle{Graficando los datos}
Para cada conjunto de datos de las inclinaciones $\ang{5}, \ang{10}, \ang{15}$:
\setbeamercolor{item projected}{bg=byzantine,fg=white}
\setbeamertemplate{enumerate items}{%
\usebeamercolor[bg]{item projected}%
\raisebox{1.5pt}{\colorbox{bg}{\color{fg}\footnotesize\insertenumlabel}}%
}
\begin{enumerate}[<+->]
\item Anota solo el valor del par de datos $(d, t)$.
\item Con una regla, traza una recta de tal manera que la línea esté muy cerca de cada dato.
\item Calcula la pendiente de esa recta.
\end{enumerate}
\pause
¿Qué nos dice la pendiente con respecto a la aceleración?
\end{frame}
\begin{frame}
\frametitle{Segunda gráfica}
Elabora una gráfica por cada ángulo de inclinación, con las variables:
\begin{table}
    \centering
    \begin{tabular}{c | c}
        Velocidad [\unit{\meter}] & Tiempo$^{2}$ [\unit{\square\second}]\\ \hline
         & \\ \hline       
         & \\ \hline       
    \end{tabular}
\end{table}
\end{frame}
\begin{frame}
\frametitle{Graficando los datos}
Maneja las variables de la forma:
\begin{figure}
    \centering
    \begin{tikzpicture}
        \draw [-stealth] (0, 0) -- (5, 0) node [above, pos=1] {\small{$t^{2} \, [\unit{\square\second}]$}};
        \draw [-stealth] (0, 0) -- (0, 3) node [left, pos=1] {\small{$v \, [\unit{\meter\per\second}]$}};
    \end{tikzpicture}
\end{figure}
Gráficamente, ¿qué sabemos de la aceleración?
\end{frame}
\begin{frame}
\frametitle{Calculando la aceleración}
Sabemos que la velocidad inicial de la canica o balín es $v_{i} = 0$, anotando el último valor de velocidad en \SI{100}{\centi\meter}, calcula la aceleración en cada ángulo de inclinación.
\end{frame}
\begin{frame}
\frametitle{Calculando la aceleración}
¿Qué nos dice el valor de la aceleración en todo el tramo del riel con respecto al ángulo de inclinación?
\end{frame}

\section{El reporte}
\frame{\tableofcontents[currentsection, hideothersubsections]}
\subsection{Preparación del reporte}

\begin{frame}
\frametitle{Entrega en equipo}
En esta ocasión, el reporte se deberá de preparar en equipo, siguiendo los puntos:
\setbeamercolor{item projected}{bg=cobalt,fg=white}
\setbeamertemplate{enumerate items}{%
\usebeamercolor[bg]{item projected}%
\raisebox{1.5pt}{\colorbox{bg}{\color{fg}\footnotesize\insertenumlabel}}%
}
\begin{enumerate}[<+->]
\item Anotar el nombre el integrante en la hoja y equipo al que pertenece.
\seti
\end{enumerate}
\end{frame}
\begin{frame}
\frametitle{Entrega en equipo}
\setbeamercolor{item projected}{bg=cobalt,fg=white}
\setbeamertemplate{enumerate items}{%
\usebeamercolor[bg]{item projected}%
\raisebox{1.5pt}{\colorbox{bg}{\color{fg}\footnotesize\insertenumlabel}}%
}
\begin{enumerate}[<+->]
\conti
\item Cada integrante debe de preparar en su cuaderno y/o libreta, escrito a mano, la tabla de valores de cada registro de distancia y tiempo, de cada inclinación
\seti
\end{enumerate}
\end{frame}
\begin{frame}
\frametitle{Entrega en equipo}
\setbeamercolor{item projected}{bg=cobalt,fg=white}
\setbeamertemplate{enumerate items}{%
\usebeamercolor[bg]{item projected}%
\raisebox{1.5pt}{\colorbox{bg}{\color{fg}\footnotesize\insertenumlabel}}%
}
\begin{enumerate}[<+->]
\conti
\item Preparar una gráfica de desplazamiento y tiempo, identificando el ángulo que ocupa para graficar, deberá de  anotar el nombre de la alumna/alumno y responder la pregunta que se indica.
\seti
\end{enumerate}
\end{frame}
\begin{frame}
\frametitle{Entrega en equipo}
\setbeamercolor{item projected}{bg=cobalt,fg=white}
\setbeamertemplate{enumerate items}{%
\usebeamercolor[bg]{item projected}%
\raisebox{1.5pt}{\colorbox{bg}{\color{fg}\footnotesize\insertenumlabel}}%
}
\begin{enumerate}[<+->]
\conti
\item Cada integrante debe de preparar la gráfica de velocidad y tiempo, identificando el ángulo que ocupa para graficar, anotando su nombre y responder la pregunta que se indica en la diapositiva.
\end{enumerate}
\end{frame}
\begin{frame}
\frametitle{Aportación individual}
Cada integrante deberá de aportar su tabla de datos (aunque sean los mismos), una gráfica de desplazamiento y tiempo, con la respuesta a la pregunta.
\\
\bigskip
Así como la gráfica de velocidad y tiempo al cuadrado, con la respuesta a la pregunta.
\end{frame}
\begin{frame}
\frametitle{Entrega en equipo}
El equipo debe de organizarse para reunir las aportaciones de cada integrante y preparar un solo archivo.
\\
\bigskip
Este archivo deberá de compartirse con cada integrante.
\end{frame}
\begin{frame}
\frametitle{Envío del reporte}
Se abrirá una asignación en Teams para que cada alumna/alumno envíe el archivo con el reporte.
\\
\bigskip
En caso de que se haya participado en la actividad, se haya enviado la aportación, pero no haya enviado el archivo del equipo, tendrá cero en la práctica.
\end{frame}
\begin{frame}
\frametitle{Envío del reporte}
Como se pide la organización en equipo, si algún integrante del equipo no envía su aportación y no aparece en el archivo, se descontará evaluación a todo el equipo.
\end{frame}
\begin{frame}
\frametitle{Envío del reporte}
La responsabilidad de trabajo es de TODO el equipo, no solo de quien organiza o propone armar el archivo.
\end{frame}
\begin{frame}
\frametitle{Plazo de envío}
Se deberá de enviar por la asignación en Teams el archivo del reporte de la Práctica 5, a más tardar el miércoles 19 de julio a las 8 pm.
\end{frame}
\begin{frame}
\frametitle{Reporte como parte de la evaluación}
Este reporte será parte de la segunda evaluación parcial.
\end{frame}

\end{document}